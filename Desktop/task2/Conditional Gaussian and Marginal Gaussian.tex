\documentclass{article}
\title{Conditional Gaussian and Marginal Gaussian}
\author{Xinghu Yao}
\date{\today}

\usepackage{geometry}
\geometry{a4paper,scale=0.8}
\usepackage{amsfonts}
\usepackage{amsmath}
\usepackage{amssymb}


\begin{document}
	\maketitle
	\section{Question}
    Given a joint Gaussian distribution $\mathcal{N}\left(\mathbf{x}|\boldsymbol{\mu},\mathbf{\Sigma}\right)$ with $\mathbf{\Lambda} \equiv \mathbf{\Sigma}^{-1}$ and
    \begin{equation}
        \mathbf{x} =\left( {\begin{array}{c}
    \mathbf{x}_a\\
    \mathbf{x}_b
    \end{array}} \right),\quad
    \boldsymbol{\mu} =\left( {\begin{array}{c}
    \boldsymbol{\mu}_a\\
    \boldsymbol{\mu}_b
    \end{array}} \right)\label{eq:joint1}
   \end{equation}
   \begin{equation}
    \mathbf{\Sigma} = \left(\begin{array}{cc}
    \mathbf{\Sigma}_{aa} & \mathbf{\Sigma}_{ab}\\
    \mathbf{\Sigma}_{ba} & \mathbf{\Sigma}_{bb}
    \end{array}\right),\quad
    \mathbf{\Lambda} =\left(\begin{array}{cc}
    \mathbf{\Lambda}_{aa} & \mathbf{\Lambda}_{ab}\\
    \mathbf{\Lambda}_{aa} & \mathbf{\Lambda}_{bb}
    \end{array}\right)\label{eq:joint2}
   \end{equation}
   Try yo derive the following conditional distribution and marginal distribution:
   \begin{align}
   p\left(\mathbf{x}_a|\mathbf{x}_b\right) &=\mathcal{N}\left(\mathbf{x}|\boldsymbol{\mu}_{a|b},\mathbf{\Lambda}_{aa}^{-1}\right)\\
   \boldsymbol{\mu}_{a|b} &= \boldsymbol{\mu}_a - \mathbf{\Lambda}_{aa}^{-1}\mathbf{\Lambda}_{ab}\mathbf{x}_b - \boldsymbol{\mu}_b.\\
   p\left(\mathbf{x}_n\right) &= \mathcal{N}\left(\mathbf{x}_a|\boldsymbol{\mu}_a,\mathbf{\Sigma}_{aa}\right).
   \end{align}
   \section{Conditional Gaussian}
   \par According to the definition of condition distribution, we have $p\left(\mathbf{x}_a|\mathbf{x}_b\right) = \frac{p\left(\mathbf{x}_a,\mathbf{x}_b\right)}{p\left(\mathbf{x}_b\right)}$. Thus, through fixing $\mathbf{x}_b$ to the observed value and normalizing the resulting expression with $p\left(\mathbf{x}_b\right)$, we can get the conditional distribution $p\left(\mathbf{x}_a|\mathbf{x}_b\right)$.
	
	We firstly consider the quadratic form in the exponent of the Gaussian distribution give by Eq.~(\ref{eq:joint1}) and Eq.~(\ref{eq:joint2}). In fact, we have
   \begin{align}
   &-\frac{1}{2}\left(\mathbf{x} - \boldsymbol{\mu}\right)^{T}\boldsymbol{\Sigma}^{-1}\left(\mathbf{x}- \boldsymbol{\mu}\right)\notag\\
   &= -\frac{1}{2}\left(\begin{array}{c}
   \mathbf{x}_a - \boldsymbol{\mu}_a\\
   \mathbf{x}_b - \boldsymbol{\mu}_b
   \end{array}\right)^T
   \left(\begin{array}{cc}
   \mathbf{\Lambda}_{aa} & \mathbf{\Lambda}_{ab}\notag\\
   \mathbf{\Lambda}_{ba} & \mathbf{\Lambda}_{bb}\end{array}\right)
   \left(\begin{array}{c}
   \mathbf{x}_a - \boldsymbol{\mu}_a\\
   \mathbf{x}_b - \boldsymbol{\mu}_b\\\end{array}\right)\notag\\
   &= -\frac{1}{2}\left[\left(\mathbf{x}_a - \boldsymbol{\mu}_a\right)^T,\left(\mathbf{x}_b - \boldsymbol{\mu}_b\right)^T\right]
   \left(\begin{array}{cc}
   \mathbf{\Lambda}_{aa} & \mathbf{\Lambda}_{ab}\\
   \mathbf{\Lambda}_{ba} & \mathbf{\Lambda}_{bb}\end{array}\right)
   \left(\begin{array}{c}
   \mathbf{x}_a - \boldsymbol{\mu}_a\\
   \mathbf{x}_b - \boldsymbol{\mu}_b\\\end{array}\right)\notag\\
   & = -\frac{1}{2}\left(\mathbf{x}_a - \boldsymbol{\mu}_a\right)^T\boldsymbol{\Lambda}_{aa}\left(\mathbf{x}_a -\boldsymbol{\mu}_a\right)-\frac{1}{2}\left(\mathbf{x}_a - \boldsymbol{\mu}_a\right)^T\boldsymbol{\Lambda}_{ab}\left(\mathbf{x}_b -\boldsymbol{\mu}_b\right)\notag\\&-\frac{1}{2}\left(\mathbf{x}_b - \boldsymbol{\mu}_b\right)^T\boldsymbol{\Lambda}_{ba}\left(\mathbf{x}_a -\boldsymbol{\mu}_a\right)
   -\frac{1}{2}\left(\mathbf{x}_b - \boldsymbol{\mu}_b\right)^T\boldsymbol{\Lambda}_{bb}\left(\mathbf{x}_b -\boldsymbol{\mu}_b\right).\label{eq:joint3}
   \end{align}
   \par We see that as a function of $\mathbf{x}_a$, this is a quadratic form, thus the conditional distribution $p\left(\mathbf{x}_a|\mathbf{x}_b\right)$ is a Gaussian distribution. Noticing that the exponent of a general Gaussian distribution $\mathcal{N}\left(\mathbf{x}|\boldsymbol{\mu},\mathbf{\Sigma}\right)$ can be written
   \begin{equation}
   -\frac{1}{2}\left(\mathbf{x} - \boldsymbol{\mu}\right)^{T}\boldsymbol{\Sigma}^{-1}\left(\mathbf{x}- \boldsymbol{\mu}\right) = -\frac{1}{2}\mathbf{x}^T\boldsymbol{\Sigma}^{-1}\mathbf{x} + \mathbf{x}^T\boldsymbol{\Sigma}^{-1}\boldsymbol{\mu}+const\label{eq:joint4}
   \end{equation} 
   where 'const' denotes terms which are independent of $\mathbf{x}$. Consider the functional dependence of Eq.~(\ref{eq:joint3}) on $\mathbf{x}_b$ in which $\mathbf{x}_a$ is regarded as a constant. The second order of $\mathbf{x}_a$ can be written
   \begin{equation}
   -\frac{1}{2}\mathbf{x}_a^T\Lambda_{aa}\mathbf{x}_{a}\label{eq:joint5}
   \end{equation}
   Comparing Eq.~(\ref{eq:joint4}) and Eq.~(\ref{eq:joint5}), we can immediately get the covariance matrix of $p\left(\mathbf{x}_a|\mathbf{x}_b\right)$ is given by
   \begin{equation}
   \mathbf{\Sigma}_{a|b}=\Lambda_{aa}^{-1}\label{eq:joint6}.
   \end{equation} 
   Now consider the linear form of $\mathbf{x}_a$ in Eq.~(\ref{eq:joint3})
   \begin{equation}
   \mathbf{x}_a^T\left\{\boldsymbol{\Lambda}_{aa}\boldsymbol{\mu}_a - \boldsymbol{\Lambda}_{ab}\left(\mathbf{x}_b - \boldsymbol{\mu}_b\right)\right\}\label{eq:joint7}
   \end{equation}
   Comparing Eq.~(\ref{eq:joint4}) Eq.~(\ref{eq:joint7}), we can get $\boldsymbol{\Sigma}^{-1}\boldsymbol{\mu}=\mathbf{x}_a^T\left\{\boldsymbol{\Lambda}_{aa}\boldsymbol{\mu}_a - \boldsymbol{\Lambda}_{ab}\left(\mathbf{x}_b - \boldsymbol{\mu}_b\right)\right\}$. Thus, we have 
   \begin{align}
   \boldsymbol{\mu}_{a|b} &= \boldsymbol{\Sigma}_{a|b}\left\{\boldsymbol{\Lambda}_{aa}\boldsymbol{\mu}_a - \boldsymbol{\Lambda}_{ab}\left(\mathbf{x}_b - \boldsymbol{\mu}_b\right)\right\}\notag\\
   &= \boldsymbol{\mu}_a - \boldsymbol{\Lambda}_{ab}^{-1}\boldsymbol{\Lambda}_{ab}\left(\mathbf{x}_b - \boldsymbol{\mu}_b\right)\label{eq:joint8}
   \end{align}
   Combing Eq.~(\ref{eq:joint6}) and Eq.~(\ref{eq:joint8}), we can get $p\left(\mathbf{x}_a|\mathbf{x}_b\right) =\mathcal{N}\left(\mathbf{x}|\boldsymbol{\mu}_{a|b},\mathbf{\Lambda}_{aa}^{-1}\right)$ where $\boldsymbol{\mu}_{a|b} = \boldsymbol{\mu}_a - \mathbf{\Lambda}_{aa}^{-1}\mathbf{\Lambda}_{ab}\mathbf{x}_b - \boldsymbol{\mu}_b.$
   \section{Marginal Gaussian}
   In fact, the Marginal Gaussian distribution can be written as
   \begin{equation}
   p\left(\mathbf{x}_a\right) = \int p\left(\mathbf{x}_a,\mathbf{x}_{b}\right)d\mathbf{x}_b\label{eq:joint12}
   \end{equation}
   Picking out those terms only involve $\mathbf{x}_b$ in the joint distribution $p\left(\mathbf{x}_a,\mathbf{x}_b\right)$, we have
   \begin{equation}
   -\frac{1}{2}\mathbf{x}_b^T\boldsymbol{\Lambda}_{bb}\mathbf{x}_b+\mathbf{x}_b^T\mathbf{m}=-\frac{1}{2}\left(\mathbf{x}_b-\boldsymbol{\Lambda}_{bb}^{-1}\mathbf{m}\right)^T\boldsymbol{\Lambda}_{bb}\left(\mathbf{x}_b-\boldsymbol{\Lambda}_{bb}^{-1}\mathbf{m}\right)+\frac{1}{2}\mathbf{m}^T\boldsymbol{\Lambda}_{bb}^{-1}\mathbf{m}\label{eq:joint13}
   \end{equation}
   where $\mathbf{m} = \boldsymbol{\Lambda}_{bb}\boldsymbol{\mu}_b - \boldsymbol{\Lambda}_{ba}\left(\mathbf{x}_a - \boldsymbol{\mu}_a\right)$ 
   Now we turn to consider the general Gaussian distribution, which is 
   \begin{equation}
   \mathcal{N}\left(\mathbf{x}|\boldsymbol{\mu},\boldsymbol{\Sigma}\right) = 
   \frac{1}{\left(2\pi\right)^{D/2}}\frac{1}{\left|\boldsymbol{\Sigma}\right|^{1/2}}\text{exp}\left\{-\frac{1}{2}\left(\mathbf{x}-\boldsymbol{\mu}\right)^T\boldsymbol{\Sigma}^{-1}\left(\mathbf{x}-\boldsymbol{\mu}\right)\right\} \label{eq:joint14}
   \end{equation}
   From Eq.~(\ref{eq:joint14}) we can see the coefficient of Gaussian distribution is independent of the mean and only governed by the determinant of the covariance matrix. Back to Eq.~{\ref{eq:joint13}, we can see the integration over $\mathbf{x}$ is as follows
   \begin{equation}
   \int\text{exp}\left\{-\frac{1}{2}\left(\mathbf{x}_b-\boldsymbol{\Lambda}_{bb}^{-1}\mathbf{m}\right)^T\boldsymbol{\Lambda}_{bb}\left(\mathbf{x}_b-\boldsymbol{\Lambda}_{bb}^{-1}\mathbf{m}\right)\right\}\text{d}\mathbf{x}_b\label{eq:15}.
   \end{equation}
   This integration is irrelevant with the mean so we can margin out $mathbf{x}_b$ after the integration.Combing the $\mathbf{x}_b^{T}\mathbf{m}$ in Eq.~(\ref{eq:joint13}) with the remaining terms from Eq.~(\ref{eq:joint3}), we have
   \begin{align}
   &\frac{1}{2}\left[\boldsymbol{\Lambda}_{bb}\boldsymbol{\mu}_b - \boldsymbol{\Lambda}_{ba}\left(\mathbf{x}_a - \boldsymbol{\mu}_a\right)\right]^T\boldsymbol{\Lambda}_{bb}^{-1}\left[\boldsymbol{\Lambda}_{bb}\boldsymbol{\mu}_b - \boldsymbol{\Lambda}_{ba}\left(\mathbf{x}_a - \boldsymbol{\mu}_a\right)\right]-\frac{1}{2}\mathbf{x}_a^T\boldsymbol{\Lambda}_{aa}\mathbf{x}_a+\mathbf{x}_a^T\left(\boldsymbol{\Lambda}_{aa}\boldsymbol{\mu}_a+\boldsymbol{\Lambda}_{ab}\boldsymbol{\mu}_b\right)+\text{const}\notag\\
   &=\frac{1}{2}\mathbf{x}_a^T\left(\boldsymbol{\Lambda}_{aa}-\boldsymbol{\Lambda}_{ab}\boldsymbol{\Lambda}_{bb}^{-1}\boldsymbol{\Lambda}_{ba}\right)\mathbf{x}_a+\mathbf{x}_a^T\left(\boldsymbol{\Lambda}_{aa}-\boldsymbol{\Lambda}_{ab}\boldsymbol{\Lambda}_{bb}^{-1}\right)^{-1}\boldsymbol{\mu}_a + \text{const}\label{eq:joint16}
   \end{align}
   where 'const' denotes quantities independent of $\mathbf{x}_a$. Comparing Eq.~(\ref{eq:joint16}) with Eq.~(\ref{eq:joint4}), we can get the covariance of $p\left(\mathbf{x}_a\right)$ is
   \begin{equation}
   \boldsymbol{\Sigma}_a = \left(\boldsymbol{\Lambda}_{aa}-\boldsymbol{\Lambda}_{ab}\boldsymbol{\Lambda}_{bb}^{-1}\boldsymbol{\Lambda}_{ba}\right)^{-1} = \boldsymbol{\Sigma}_{aa}\label{eq:joint17}
   \end{equation} 
   and the mean of $p\left(\mathbf{x}_a\right)$ is given by
   \begin{equation}
   \boldsymbol{\mu}_{a}=\boldsymbol{\Sigma}_{a}\left(\boldsymbol{\Lambda}_{aa}-\boldsymbol{\Lambda}_{ab}\boldsymbol{\Lambda}_{bb}^{-1}\boldsymbol{\Lambda}_{ba}\right)\boldsymbol{\mu}_{a}\label{eq:joint18}
   \end{equation}
   We can summarized Eq.~(\ref{eq:joint17}) and Eq.~(\ref{eq:joint18}) as
   \begin{equation}
   p\left(\mathbf{x}_a\right) = \mathcal{N}\left(\mathbf{x}_a|\boldsymbol{\mu}_a,\boldsymbol{\Sigma}_{aa}\right)
   \end{equation}
\end{document}
