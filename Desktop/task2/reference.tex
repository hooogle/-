\documentclass{article}
\title{The Derivation of Probabilistic PCA}
\author{YaoXinghu}
\date{\today}

\usepackage{geometry}
\geometry{a4paper,scale=0.8}
\usepackage{amsfonts}
\usepackage{amsmath}
\begin{document}
	\maketitle
	\section{previous derivation}
	Given the dataset $D=\{\mathbf{x_i}\}_{i=1}^N, \mathbf x_i \in \mathbb R^d $, we introduce a explicit latent variable $\mathbf{z}$ corresponding to a subspace with a smaller dimensionality and the prior distribution over $\mathbf z$ is defined as a Gaussian distribution $p\left(\mathbf z \right)$:
	\begin{equation}
		 p\left(\mathbf z \right) = \mathcal N \left(\mathbf z|\mathbf 0, \mathbf I \right)\label{eq:prior}
	\end{equation}
     Here we define a mapping from latent space to data space $\mathbf z \rightarrow \mathbf x$:
     \begin{equation}
     	\mathbf x = \mathbf W \mathbf z + \boldsymbol \mu + \boldsymbol \epsilon
     \end{equation}
     where the noise Gaussian variable $\boldsymbol \epsilon$ is given by zero-mean and convariance $ \sigma^2 \mathbf I$, so the conditional distribution $p\left(\mathbf x|\mathbf z \right)$ is again Gaussian:
     \begin{equation}
     	p\left(\mathbf x|\mathbf z \right) = \mathcal N \left(\mathbf x|\mathbf W \mathbf z + \boldsymbol \mu, \sigma^2 \mathbf I \right)\label{eq:conditional}
     \end{equation}
	\section{maximum likelihood}
	To determine the parameters of the generative model, $ \theta = \{ \mathbf W, \boldsymbol \mu, \sigma^2 \}$, we utilize the maximum likelihood:
	\begin{equation}
		L\left(\theta\right) = \mbox{log}\; p\left(\mathbf x| \theta \right)
	\end{equation}
	And the marginal distribution can be rewritten as:
	\begin{align}
       \mbox{log}\; p\left(\mathbf x| \theta \right) &= \mbox{log}\; \int_{\mathbf z}^{} p\left(\mathbf x, \mathbf z| \theta \right)\notag\\
       & = \mbox{log}\; \int_{\mathbf z}^{} p\left(\mathbf x|\mathbf z, \theta \right)p\left(\mathbf z \right)\label{eq:marginal}
	\end{align}
	According to the Eq.~(\ref{eq:prior}) and Eq.~(\ref{eq:conditional}), the joint distribution $p\left(\mathbf x, \mathbf z| \theta \right)$ can be written as:
	\begin{small}
	\begin{align}
		& p\left(\mathbf x, \mathbf z| \theta \right) = p\left(\mathbf x|\mathbf z, \theta \right)p\left(\mathbf z \right)\notag\\
		&\propto \mbox{exp}\left(-\frac{\left(\mathbf x-\mathbf W \mathbf z - \boldsymbol{\mu}\right)^{\top}\left(\mathbf x-\mathbf W \mathbf z - \boldsymbol{\mu}\right)}{2\sigma^2}\right)\mbox{exp}\left(-\frac{\mathbf z^\top \mathbf z}{2}\right)\notag\\
		&\propto \mbox{exp}\left(-\frac{\mathbf x^\top \mathbf x - \mathbf x^\top \mathbf W \mathbf z - \mathbf x^\top \boldsymbol \mu - \mathbf z^\top \mathbf W^\top \mathbf x + \mathbf z^\top \mathbf W^\top \mathbf W \mathbf z + \mathbf z^\top \mathbf W^\top \boldsymbol \mu - \boldsymbol \mu^\top \mathbf x + \boldsymbol \mu^\top \mathbf W \mathbf z + \boldsymbol \mu^\top \boldsymbol \mu}{2\sigma^2}-\frac{\mathbf z^\top \mathbf z}{2}\right)\notag\\
		& \propto \mbox{exp} \left( -\frac{1}{2}\left( \left(\mathbf x^\top - \boldsymbol \mu^\top\right)\left(\frac{\mathbf I}{\sigma^2}\right)\mathbf x - \left(\mathbf x^\top - \boldsymbol \mu^\top \right)\left(\frac{\mathbf I}{\sigma^2}\right)\boldsymbol \mu -\left(\mathbf x^\top - \boldsymbol \mu^\top \right)\left(\frac{\mathbf W}{\sigma^2}\right)\mathbf z -\mathbf z^\top \left(\frac{\mathbf W^\top}{\sigma^2}\right)\left(\mathbf x - \boldsymbol \mu\right)+\mathbf z^\top\left(\frac{\mathbf W^\top\mathbf W}{\sigma^2} + \mathbf I\right)\mathbf z\right) \right)	\notag	\\
		& \propto \mbox{exp} \left( -\frac{1}{2}\left( \left(\mathbf x^\top - \boldsymbol \mu^\top\right)\left(\frac{\mathbf I}{\sigma^2}\right)\left(\mathbf x - \boldsymbol \mu \right) -\left(\mathbf x^\top - \boldsymbol \mu^\top \right)\left(\frac{\mathbf W}{\sigma^2}\right)\mathbf z -\mathbf z^\top \left(\frac{\mathbf W^\top}{\sigma^2}\right)\left(\mathbf x - \boldsymbol \mu\right)+\mathbf z^\top\left(\frac{\mathbf W^\top\mathbf W}{\sigma^2} + \mathbf I\right)\mathbf z\right) \right)	\notag\\
		& \propto \mbox{exp} \left(-\frac{1}{2}\left(\left[\mathbf z^\top \;
		\left(\mathbf x - \boldsymbol \mu\right)^\top\right]\begin{bmatrix}
		\frac{\mathbf W^\top\mathbf W}{\sigma^2} + \mathbf I     & -\frac{\mathbf W^\top}{\sigma^2}      \\
		-\frac{\mathbf W}{\sigma^2}  & \frac{\mathbf I}{\sigma^2}
		\end{bmatrix}\begin{bmatrix}
		\mathbf z\\
		\mathbf x - \boldsymbol \mu
		\end{bmatrix}\right)\right)	\label{eq:joint_dist}
	\end{align}
	\end{small}
From the Eq.~(\ref{eq:joint_dist}), we can see that the joint distribution  actually is a Gaussian over the joint variable $\left[\mathbf z;\mathbf x\right]$:
\begin{equation}
    p\left(\mathbf x, \mathbf z| \theta \right) = \mathcal N \left(\begin{bmatrix}
    \mathbf z\\
    \mathbf x
    \end{bmatrix}|\begin{bmatrix}
    \mathbf 0 \\
    \boldsymbol \mu
    \end{bmatrix}, \boldsymbol \Sigma_{\mathbf z, \mathbf x}\right)
\end{equation}

where
\begin{equation}
	\boldsymbol \Sigma_{\mathbf z, \mathbf x} = \begin{bmatrix}
	\boldsymbol \Sigma_{\mathbf z \mathbf z} & \boldsymbol \Sigma_{\mathbf z \mathbf x}\\
	 \boldsymbol \Sigma_{\mathbf x \mathbf z} & \boldsymbol \Sigma_{\mathbf x \mathbf x}
	\end{bmatrix} = \begin{bmatrix}
	\frac{\mathbf W^\top\mathbf W}{\sigma^2} + \mathbf I     & -\frac{\mathbf W^\top}{\sigma^2}      \\
	-\frac{\mathbf W}{\sigma^2}  & \frac{\mathbf I}{\sigma^2} \label{eq:covariance}
	\end{bmatrix}^{-1}
\end{equation}
Using the partitioned inversion lemma(Sherman-Morrison-Woodbury formula), Eq.~(\ref{eq:covariance}) can be convertd to:
\begin{equation}
\boldsymbol \Sigma_{\mathbf z, \mathbf x} = \begin{bmatrix}
\mathbf I & \mathbf W^\top \\
\mathbf W & \mathbf W \mathbf W^\top + \sigma^2 \mathbf I
\end{bmatrix}
\end{equation}

So the marginal distribution is given by:
\begin{equation}
p\left(\mathbf x| \theta \right) = \mathcal N \left(\mathbf x|\boldsymbol \mu, \mathbf W \mathbf W^\top + \sigma^2 \mathbf I\right)
\end{equation}
thus the likelihood function can be rewritten as:
\begin{align}
	L\left(\theta\right) &= \mbox{log}\; p\left(\mathbf x| \theta \right) = \sum_{n=1}^{N} \mbox{log} \; p\left(\mathrm x_n| \theta \right)\\
	&= -\frac{N}{2}\mbox{log} \left|\mathbf W \mathbf W^\top + \sigma^2 \mathbf I\right| - \frac{1}{2}\sum_{n=1}^{N} \left(\mathrm x_n - \boldsymbol \mu\right)^\top \left(\mathbf W \mathbf W^\top + \sigma^2 \mathbf I\right) \left(\mathrm x_n - \boldsymbol \mu\right) + \mbox{constc}
\end{align}

\end{document}
